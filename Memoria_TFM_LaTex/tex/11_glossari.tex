% #########################################################################
% ##                       FITXER: 11_glossari_entries.tex                ##
% ##         Contingut: Definicions de glossari i acrònims per \gls      ##
% #########################################################################

% % Categories de glossaris
% \newglossary[markov]{markov}{mkv}{nmk}{Variables de Markov}
% \newglossary[arima]{arima}{ari}{nari}{Paràmetres ARIMA i SARIMA}
% \newglossary[lstm]{lstm}{lst}{nlst}{Paràmetres i símbols LSTM}

% ------------------------------------------------------------
% ACRÒNIMS
% ------------------------------------------------------------

\newacronym{uab}{UAB}{Universitat Autònoma de Barcelona}
\newacronym{smc}{SMC}{Servei Meteorològic de Catalunya}
\newacronym{xema}{XEMA}{Xarxa d'Estacions Meteorològiques Automàtiques}
\newacronym{ml}{ML}{Aprenentatge automàtic (Machine Learning)}
\newacronym{arima}{ARIMA}{Model autoregressiu integrat de mitjana mòbil}
\newacronym{sarima}{SARIMA}{Seasonal ARIMA}
\newacronym{lstm}{LSTM}{Long-Short Term Memory}
\newacronym{rnn}{RNN}{Recurrent Neural Network}
\newacronym{rmse}{RMSE}{Root Mean Squared Error}
\newacronym{mse}{MSE}{Mean Squared Error}
\newacronym{mae}{MAE}{Mean Absolute Error}

% ------------------------------------------------------------
% VARIABLES I SÍMBOLS
% ------------------------------------------------------------

% MARKOV

\newglossaryentry{pij}{
  name = $P_{ij}$,
  description = {Probabilitat de transició de l’estat $i$ a $j$ en un model de Markov}
}

% ARIMA

\newglossaryentry{yt}{
    name={\ensuremath{y_t}},
    description={Valor observat de la sèrie temporal en el temps \ensuremath{t}.}
}

\newglossaryentry{c}{
    name={\ensuremath{c}},
    description={Constant o terme independent del model.}
}

\newglossaryentry{phi}{
    name={\ensuremath{\phi_i}},
    description={Coeficients del component autoregressiu (AR), que ponderen la influència dels valors passats \ensuremath{y_{t-i}}.}
}

\newglossaryentry{theta}{
    name={\ensuremath{\theta_j}},
    description={Coeficients del component de mitjana mòbil (MA), que ponderen els errors passats \ensuremath{\varepsilon_{t-j}}.}
}

\newglossaryentry{eps}{
    name={\ensuremath{\varepsilon_t}},
    description={Soroll blanc: error aleatori amb mitjana zero i no correlacionat.}
}

\newglossaryentry{epsj}{
    name={\ensuremath{\varepsilon_{t-j}}},
    description={Error de predicció comès en l’instant \ensuremath{t-j}, és a dir, la diferència entre el valor observat i el predit.}
}

\newglossaryentry{B}{
    name={\ensuremath{B}},
    description={Operador de retard (\textit{backshift}): \ensuremath{B y_t = y_{t-1}}. Permet expressar diferències i components estacionals.}
}

\newglossaryentry{phipB}{
    name={\ensuremath{\phi_p(B)}},
    description={Polinomi que representa el component AR no estacional d’ordre \ensuremath{p}, en funció de l’operador \ensuremath{B}.}
}

\newglossaryentry{thetaqB}{
    name={\ensuremath{\theta_q(B)}},
    description={Polinomi del component MA no estacional d’ordre \ensuremath{q}, en funció de \ensuremath{B}.}
}

\newglossaryentry{PhiPB}{
    name={\ensuremath{\Phi_P(B^s)}},
    description={Polinomi del component AR estacional, d’ordre \ensuremath{P} i període \ensuremath{s}.}
}

\newglossaryentry{ThetaQB}{
    name={\ensuremath{\Theta_Q(B^s)}},
    description={Polinomi del component MA estacional, d’ordre \ensuremath{Q} i període \ensuremath{s}.}
}

\newglossaryentry{d}{
    name={\ensuremath{d}},
    description={Nombre de diferenciacions no estacionals aplicades a la sèrie.}
}

\newglossaryentry{D}{
    name={\ensuremath{D}},
    description={Nombre de diferenciacions estacionals aplicades a la sèrie.}
}

\newglossaryentry{s}{
    name={\ensuremath{s}},
    description={Període d’estacionalitat (p. ex., \ensuremath{s=24} per cicles diürns en dades horàries).}
}



% LSTM

\newglossaryentry{xt}{
    name={$x_t$},
    description={Entrada de la cèl·lula en el pas temporal $t$}
}

\newglossaryentry{htm1}{
    name={$h_{t-1}$},
    description={Sortida (estat ocult) de la cèl·lula en el pas anterior}
}

\newglossaryentry{ctm1}{
    name={$C_{t-1}$},
    description={Estat intern (memòria) de la cèl·lula en el pas anterior}
}

\newglossaryentry{ct}{
    name={$C_t$},
    description={Estat intern (memòria) actualitzat en el pas $t$}
}

\newglossaryentry{ht}{
    name={$h_t$},
    description={Sortida de la cèl·lula en el pas actual}
}

\newglossaryentry{ft}{
    name={$f_t$},
    description={Porta d’oblit: controla quina part de la memòria anterior es descarta}
}

\newglossaryentry{it}{
    name={$i_t$},
    description={Porta d’entrada: regula quina nova informació s’incorpora}
}

\newglossaryentry{ot}{
    name={$o_t$},
    description={Porta de sortida: determina quina part de la memòria es propaga com a sortida}
}

\newglossaryentry{ctilda}{
    name={$\tilde{C}_t$},
    description={Cèl·lula candidata: nova informació proposada per ser incorporada a la memòria}
}

\newglossaryentry{wstar}{
    name={$W_*$},
    description={Matriu de pesos aplicada a l’entrada $x_t$ per a cada porta ($* = f, i, o, c$)}
}

\newglossaryentry{ustar}{
    name={$U_*$},
    description={Matriu de pesos aplicada a l’estat ocult $h_{t-1}$ per a cada porta}
}

\newglossaryentry{bstar}{
    name={$b_*$},
    description={Termes de biaix de cada porta}
}

\newglossaryentry{sigmoid}{
    name={$\sigma(\cdot)$},
    description={Funció sigmoide, amb sortida a l’interval $[0, 1]$}
}

\newglossaryentry{tanh}{
    name={$\tanh(\cdot)$},
    description={Funció tangenta hiperbòlica, amb sortida a l’interval $[-1, 1]$}
}



